%--- Hooks ------------------------------------------------------------------------------
\section{Hooks ---\\escape hatches \\let you``step outside'' React}

%--- React Basics ------------------------------------------------------------------------------
\begin{frame}[fragile] \frametitle{React Basics}
Life of a component
\begin{itemize}
  \item mount
  \item update
  \item unmount
\end{itemize}
Update --- render function
\begin{itemize}
  \item computes the react DOM tree from \code{props} and state
  \item must be pure
  \item no control of when or how often it is called (console.log will be messy)
\end{itemize}
\vspace{5mm}
Event handlers
\begin{itemize}
  \item call is triggered by user actions
  \item may change state
\end{itemize}
\end{frame}

%--- Hooks ------------------------------------------------------------------------------
\begin{frame}[fragile] \frametitle{Hooks}
\begin{itemize}
  \item escapes the limitations of ``React basics''
  \item hooks into the internals of react
  \item covers more use cases than react basics
  \item named \code{useSomething()}
  \item you have seen \code{useState()} and \code{useRef()}
  \item there are many more
\end{itemize}
\vspace{5mm}
Rules of hooks:
\begin{itemize}
  \item called from render code directly, or other hooks
  \item must be called in the same order every re-render
  \item only call from top level code (not insida conditions or loops)
\end{itemize}
\end{frame}

%--- Effects ------------------------------------------------------------------------------
\begin{frame}[fragile] \frametitle{Effects}
``side effects that are caused by rendering itself, rather than by a particular event''
\begin{itemize}
  \item example: fetch data from a server when the component is viewed
\end{itemize}
\vspace{5mm}
\begin{CodeBox}{}
import { useEffect } from 'react';

function MyComponent() {
  useEffect(() => {
    // Code here will run after *every* render
  });
  return <div />;
}
\end{CodeBox}
\end{frame}

%--- Effects ------------------------------------------------------------------------------
\begin{frame}[fragile] \frametitle{Effects}
\begin{CodeBox}{}
useEffect(() => {
  // initialisation_code
  return 
    () => {/* clean_up_code */}
  }, [list of dependencies]);
\end{CodeBox}
\vspace{5mm}
\begin{itemize}
  \item \code{initialisation_code} is always followed by \code{clean_up_code}
  \item \code{initialisation_code} is run after component is mounted in the DOM, and when a variable in the dependency list is updated
  \item \code{clean_up_code} is run when the component is removed from the DOM and before \code{initialisation_code} is run due to a value change
  \item \code{list of dependencies} --- must be all \code{props}/state read
\end{itemize}
\end{frame}

%--- Effects - Ticker------------------------------------------------------------------------------
\begin{frame}[fragile] \frametitle{Effects in Ticker example}
\begin{CodeBox}{}
import { useEffect, useState } from 'react';

function Ticker({delay}) {
  const [cnt, setCnt] = useState(0);
  
  useEffect(() => {
    const id = setInterval(_=> setCnt(cnt => cnt+1), delay);
    return () => {clearInterval(id)}
  }, [delay]);
  
  return (<p> ticks: {cnt} </p>);
}
\end{CodeBox}
\end{frame}

%--- Effects - Ticker run------------------------------------------------------------------------------
\begin{frame}[fragile] \frametitle{Effects - Running Ticker}
component mounts:
\begin{itemize}
  \item run \code{initialisation\_code} (closure 1)
\end{itemize}
\code{delay} is changed:
\begin{itemize}
  \item run \code{clean\_up\_code} (closure 1)
  \item run \code{initialisation\_code} (closure 2)
\end{itemize}
\code{delay} is changed:
\begin{itemize}
  \item run \code{clean\_up\_code} (closure 2)
  \item run \code{initialisation\_code} (closure 3)
\end{itemize}
\ldots
component is deleted:
\begin{itemize}
  \item run \code{clean\_up\_code} (closure 3)
\end{itemize}
\end{frame}

%--- Effects - Development mode ------------------------------------------------------------------------------
\begin{frame}[fragile] \frametitle{Effects - Development mode}
In development mode:
\begin{itemize}
  \item components are mounted twice (and deleted once)
  \item stress test the clean up code in effects
  \item will mess upp any \code{console.log}
\end{itemize}
\end{frame}

%--- Context ------------------------------------------------------------------------------
\section{Context ---\\broadcasting in a tree}

%--- Context, broadcasting in a tree ------------------------------------------------------------------------------
\begin{frame}[fragile] \frametitle{Context, broadcasting in a tree}
\begin{itemize}
  \item \code{props} are convenient when communication components are few and close
  \item do not scale
  \item context can broadcast a value to all components in a tree
  \item three steps:
  \begin{itemize}
    \item create the context
    \item provide the value for a subtree
    \item us the value in a component
   \end{itemize}
\end{itemize}
\end{frame}

%--- Context, create ------------------------------------------------------------------------------
\begin{frame}[fragile] \frametitle{Context, create}

\begin{CodeBox}{Create the context in a separate file: MyContext.js}
import { createContext } from "react";

export const MyContext = createContext('default value');
\end{CodeBox}
\end{frame}

%--- Context, provide ------------------------------------------------------------------------------
\begin{frame}[fragile] \frametitle{Context, provide a value}
\begin{CodeBox}{Provide a value in a JSX expression}
import { MyContext } from 'MyContext';

function App() {
  return (
    <MyContext.Provider value={'top level'}>
      <ViewMyContext />
      <MyContext.Provider value={'subtree'}>
        <ViewMyContext />
      </MyContext.Provider>
    </MyContext.Provider>
  );
}
\end{CodeBox}
\end{frame}

%--- Context, use ------------------------------------------------------------------------------
\begin{frame}[fragile] \frametitle{Context, use}

\begin{CodeBox}{Use the context in a component}
import { useContext } from 'react';
import { MyContext } from './MyContext.js';

export function ViewMyContext() {
  const myValue = useContext(MyContext);
  return <p>context is: {myValue}</p>
}
\end{CodeBox}
\end{frame}

%--- Reducer ------------------------------------------------------------------------------
\section{Reducers ---\\Separating state logic}

\begin{frame}[fragile] \frametitle{Reducers}
Using handelrs:
\begin{itemize}
  \item \code{nextState = handler(event, currentState)}
  \item mixing UI code with state logic
  \item hard to reuse state logic in different UI components
\end{itemize}
\vspace{5mm}
Reducers
\begin{itemize}
  \item separate view (render) from state logic (reducer) 
  \item UI code emits actions, reducer manage state
  \item pure function: \code{nextState = reducer(action, currentState)}
  \item easy to test state logic
  \item easy to reuse state logic
  \item clear which operations are allowed on the state
\end{itemize}
\end{frame}

%--- Action ------------------------------------------------------------------------------
\begin{frame}[fragile] \frametitle{Action --- a plain JavaScript object}
\begin{itemize}
  \item \code{type} property, which action to execute on the state
  \item contains all information needed to perform the action
\end{itemize}
\vspace{5mm}
\begin{CodeBox}{}
{
  type: "Add TODO",
  task: { id, text, moreData}
}
\end{CodeBox}
\end{frame}

%--- Reducer ------------------------------------------------------------------------------
\begin{frame}[fragile] \frametitle{Reducer --- a pure JavaScript function}
\begin{CodeBox}{}
function tasksReducer(tasks, action) {
  switch (action.type) {
    case 'Delete TODO': {
      return tasks.filter((t) => t.id !== action.id);
    }
    // more actions
    default: {
      throw Error('Unknown action: ' + action.type);
    }
  }
}
\end{CodeBox}
\end{frame}

%--- useReducer ------------------------------------------------------------------------------
\begin{frame}[fragile] \frametitle{useReducer}
\begin{CodeBox}{}
function MyComponent() {
  const [tasks, dispatch] = useReducer(tasksReducer, initialTasks);
  
  function handleDeleteTask(task) {
    dispatch({
      type: 'Delete TODO',
      id: task.id,
    });
  }
  
  return <button onChange={handleDeleteTask}>delete</button>}
}
\end{CodeBox}
\end{frame}

%--- TODO ------------------------------------------------------------------------------
\section{Redux ---\\ combining context and reducers}
\begin{frame}[fragile] \frametitle{Redux}
\begin{itemize}
  \item first introduced by facebook (called flux then)
  \item soved the never ending message count bug
  \item simplifies component dependencies
  \item component $\rightarrow$ store $\rightarrow$ component
  \item store --- singel global app state
\end{itemize}
\end{frame}

%--- Redux ------------------------------------------------------------------------------
\begin{frame}[fragile] \frametitle{Redux --- TasksContext.js}
\begin{CodeBox}{}
const TasksContext = createContext(null);
const TasksDispatchContext = createContext(null);

export function useTasks() {
  return useContext(TasksContext);
}

export function useTasksDispatch() {
  return useContext(TasksDispatchContext);
}

function tasksReducer(tasks, action) {
  // your code
}
\end{CodeBox}
\end{frame}

%--- Redux ------------------------------------------------------------------------------
\begin{frame}[fragile] \frametitle{Redux --- TasksContext.js}
\begin{CodeBox}{}
export function TasksProvider({ children }) {
  const [tasks, dispatch] = useReducer(
    tasksReducer,
    initialTasks
  );

  return (
    <TasksContext.Provider value={tasks}>
      <TasksDispatchContext.Provider value={dispatch}>
        {children}
      </TasksDispatchContext.Provider>
    </TasksContext.Provider>
  );
}
\end{CodeBox}
\end{frame}

%--- Redux ------------------------------------------------------------------------------
\begin{frame}[fragile] \frametitle{Redux --- App.js}
\begin{CodeBox}{}
function App() {
  return (
    <TasksProvider>
      <MyAppComponents />
    </TasksProvider>
  );
}
\end{CodeBox}
\end{frame}

%--- Redux ------------------------------------------------------------------------------
\begin{frame}[fragile] \frametitle{Redux --- TaskList.js}
\begin{CodeBox}{}
function TaskList() {
  const tasks = useTasks();
  const dispatch = useTasksDispatch();

  function handleDelete(task){
        dispatch({
            type: 'Delete TODO',
            task
  });}
  
  return (<>tasks.map(todo => (
               <ViewTODO onDelete={handleDelete} data={task} />)
             </>);
}
\end{CodeBox}
\end{frame}
