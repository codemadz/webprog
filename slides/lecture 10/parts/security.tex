%--- Security ------------------------------------------------------------------------------
\section{Security}
%--- Security ------------------------------------------------------------------------------
\begin{frame}[fragile] \frametitle{Security}
Many parts of internet was designed based on trust
\begin{itemize}
  \item few users in the 70s
  \item everyone knew eachother
  \item example: smtp, anyone can send a mail from \emph{Santa Clause}
\end{itemize}
\vspace{5mm}
This mindset still affects the internet.
\end{frame}

%--- Web Page Composition ------------------------------------------------------------------------------
\begin{frame}[fragile] \frametitle{Web Page Composition}
A web page is a composition of resources from different places:
\begin{itemize}
  \item your own JavaScript code
  \item JavaScript code from other sources
  \item images and css from even more sources
\end{itemize}
\vspace{5mm}
Can you trust all pats of your program?
\vspace{5mm}
\\Do you give twitter access to your data when you use bootstrap?
\end{frame}

%--- Isolation ------------------------------------------------------------------------------
\begin{frame}[fragile] \frametitle{Isolation}
No harm can be done if everything is isolated.
\vspace{5mm}
\\The backend can never trust the frontend!
\vspace{5mm}
\\The web browser isolates the resources of a web page:
\begin{itemize}
  \item prevent access to cookies and localstorage
  \item prevent http communication
\end{itemize}
\vspace{5mm}
Which resources can trust each-other?
\end{frame}

%--- Same Origin Policy ------------------------------------------------------------------------------
\begin{frame}[fragile] \frametitle{Same Origin Policy}
Same Origin if the same:
\begin{itemize}
  \item URI scheme, e.g. \emph{https}
  \item host name, e.g. \emph{cs.lth.se}
  \item port, e.g. \emph{443}
\end{itemize}
\vspace{5mm}
Resources from the same origin is assumed to trust each-outer and can share data.
\end{frame}

%--- Cross Cite Scripting ------------------------------------------------------------------------------
\section{Cross Cite Scripting}
%--- Cross Cite Scripting ------------------------------------------------------------------------------
\begin{frame}[fragile] \frametitle{Cross Cite Scripting}
To prevent isolation:
\begin{itemize}
  \item malicious code want to appear to come from your origin
  \item uses vulnerabilities to \emph{inject code} into content delivered from a trusted source
\end{itemize}
\vspace{5mm}
This strategy is called \emph{Cross Site Scripting}
\end{frame}

%--- Code Injection ------------------------------------------------------------------------------
\begin{frame}[fragile] \frametitle{Code Injection}
One way to inject code:
\begin{itemize}
  \item leverage on user feedback messages:
  \item enter html with \code{<script>} tags in a form field
  \item submit the form
  \item if the original text is part of the confirmation/error message:
  \begin{itemize}
    \item the \code{script} tag is added to the DOM
    \item it is executed as part of the application, the \emph{same origin}
    \item it have full access to data, inluding authentication tokens
    \item full access to any REST-api
    \item authentication headers will be automatically added to any http request
  \end{itemize}
\end{itemize}
\end{frame}

%--- Prevent Code Injection ------------------------------------------------------------------------------
\begin{frame}[fragile] \frametitle{Prevent Code Injection}
\begin{itemize}
  \item only trust data from a known source that is signed (encrypted channel)
  \item all other data must be escaped or sanitised
  \item commonly done automatically by frameworks
  \item check the documentation
  \item react:
  \begin{itemize}
    \item escapes all stings form JavaScript in JSX: 
    \begin
    {CodeBox}{}
    function MyComponent() {
      const param = useParam();
      return <p>could not find user {param.user}</p>;
    }
    \end{CodeBox}
  \end{itemize}
\end{itemize}
\end{frame}
